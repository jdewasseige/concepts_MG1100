\section{Fluid mechanics}
\subsection{Mass conservation}
The \emph{equation of mass conservation} for an incompressible
fluid in the stationary case is the following
\[ \frac{\partial p}{\partial t} + \nabla \cdot (\rho \, \vec{v}) = 0 \]
It relates its density $\rho$ and its flow velocity $\vec{v}$.

\subsection{Viscosity}
The viscosity $\eta$ of a fluid is a measure of its resistance 
to gradual deformation by shear stress or tensile stress.
Its unit is the $\si[inter-unit-separator=$\cdot$]{\pascal\second}$.

To find the viscosity of a fluid, we can do the following experiment.
We get a sphere of known radius $r_1$ and mass $m_1$,
let $\rho_1 = m_1/V_1$ be its density
where $V_1$ is its volume deduced from $r_1$.
A cylinder of height $h$ and cross-section radius $r_2$
containing a fluid of known mass $m_2$,
let $\rho_2 = m_2/V_2$ be the fluid density
where $V_2$ is deduced from $r_2$ and $h$.

Mark with tape a starting point about $2\si{\centi\metre}$
below the surface of the liquid (in this way the sphere can
reach terminal velocity before we begin taking measurements).
The ending point should be marked about $5\si{\centi\metre}$
from the bottom.
Measure the distance $d$ between the two tapes.
Drop the sphere in the cylinder and measure $t$
as the time it takes to travel between the two tapes.
We now have the average velocity $v = d/t$ of the sphere.
The viscosity $\eta$ is found using the following formula
\[ \eta = \frac{2 \Delta\rho \, g \, r_1^2}{9 \, v} \]
where $\Delta\rho = \rho_1 - \rho_2$ 
and $g$ is the acceleration of gravity.
